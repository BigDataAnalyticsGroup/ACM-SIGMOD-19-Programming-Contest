% vim: set formatoptions-=a: set nospell:
\documentclass[xcolor=dvipsnames,british]{beamer}
\usetheme{metropolis}

% vim: set nospell: set formatoptions-=a:
\usepackage[utf8]{inputenc}         % special chars
\usepackage[]{microtype}            % general appearance of text
\usepackage[babel]{csquotes}        % context-sensitive quotation
\usepackage[dvipsnames]{xcolor}     % some predefined colors
\usepackage[plain]{fancyref}        % provides \fref and \Fref for referencing
\usepackage[T1]{fontenc}
\usepackage{algorithm}
\usepackage{algpseudocode}          % pseudo code with algorithmicx environment
\usepackage{aliascnt}               % theorem environments that can be referenced
\usepackage{amsmath}
\usepackage{amssymb}                % Math symbols
\usepackage{amsthm}                 % Theorem typesetting
\usepackage{babel}                  % typographical and hyphenation rules
\usepackage{bbm}                    % Math symbols
\usepackage{booktabs}               % adds some rulers (toprule, midrule, bottomrule)
\usepackage{datetime}               % provides \printdate and various date formats
\usepackage{emptypage}              % Suppress page numbers and headings on empty pages
%\usepackage{enumitem}               % modifiable enumerations
\usepackage{fancyhdr}
\usepackage{float}                  % Introduces the boxed float, the ruled float and the plaintop float
\usepackage{mdframed}               % framebox environments
\usepackage{geometry}
\usepackage{hyperref}               % for URLs and other links
\usepackage{isodate}                % date formatting
\usepackage{lastpage}               % for \pageref{LastPage}
\usepackage{letltxmacro}            % provides \LetLtxMacro to override commands safely
\usepackage{lipsum}                 % print Lorem ipsum with \lipsum
\usepackage{listings}
\usepackage{lmodern}                % Latin Modern Font
\usepackage{lstautogobble}          % smart indent, no trailing newline
\usepackage{makecell}               % lets one custimize tabulars
\usepackage{marvosym}               % euro symbol
\usepackage{marvosym}               % more symbols
\usepackage{ifsym}                  % geometric shapes
\usepackage{mathtools}              % math tools for amsmath
\usepackage{multicol}               % for multicol environment and \twocolumn pages
\usepackage{multirow}               % similar to multicolumn, for tabular environments
\usepackage{stmaryrd}               % Math symbols
\usepackage{subcaption}             % Used for subcaptions and subfigures.
\usepackage{subscript}              % \textsubscript (like textsuperscript)
\usepackage{tabularx}               % highly customizable tabular
\usepackage{tabu}                   % smart tabular (also for math-mode)
\usepackage{textcomp}               % more symbols
\usepackage{tikz}
%\usepackage{titlesec}               % customize chapter/section title formatting
%\usepackage{tocloft}                % allows customizing the ToC
\usepackage{varwidth}               % like minipage, but fits its content
\usepackage{wasysym}                % more symbols
\usepackage{wrapfig}                % wrapfigure
\usepackage{xspace}                 % intelligent trailing white-space

\usetikzlibrary{
    arrows,
    arrows.meta,
    backgrounds,
    calc,
    chains,
    decorations,
    decorations.pathreplacing,
    matrix,
    positioning,
    scopes,
    shadows,
    shapes,
    shapes.multipart,
}


%===============================================================================
%
%   Packages that are useful, but should be used by the parent project directly
%
%===============================================================================

% gemoetry - Modify document dimensions

% vim: set nospell:

% theorem environments
%\newtheorem{theorem}{Theorem}
%\newaliascnt{lemma}{theorem}
%\newtheorem{lemma}[lemma]{Lemma}
%\aliascntresetthe{lemma}

% references
\AtBeginDocument{
    \def\equationautorefname{Equation}
    \def\figureautorefname{Figure}
    \def\subfigureautorefname{Figure}
    \def\defnautorefname{Definition}
    \def\Itemautorefname{Item}
    \def\tableautorefname{Table}
    \def\sectionautorefname{Section}
    \def\subsectionautorefname{Section}
    \def\subsubsectionautorefname{Section}
    \def\chapterautorefname{Chapter}
    \def\partautorefname{Part}
    \def\algorithmautorefname{Algorithm}
    \def\lemmaautorefname{Lemma}
    \def\theoremautorefname{Theorem}
    \def\definitionautorefname{Definition}
}

% tabular row height
\renewcommand{\arraystretch}{1.5}

% vim: set nospell:

\newcommand{\TODO}[1]{\textbf{\textcolor{red}{TODO: #1}}}
\newcommand{\Sub}[1]{\textsubscript{#1}\xspace}
\newcommand{\Sup}[1]{\textsuperscript{#1}\xspace}

\newcommand{\Reg}{\Sup{\textregistered}}
\newcommand{\Copy}{\Sup{\textcopyright}}
\newcommand{\TM}{\Sup{\texttrademark}}

% Multiline cell inside a tabular
\newcommand{\multiline}[2][c]{%
  \begin{tabular}[#1]{@{}l@{}}#2\end{tabular}}

% vim: set nospell:

\lstset{
  autogobble=true,
  numbers=left,
  numberstyle=\footnotesize\ttfamily,
  numbersep=8pt,
  xleftmargin=3em,
  columns=fixed,
  basicstyle=\small\ttfamily,
  keywordstyle=\color{Blue}\bfseries,
  commentstyle=\color{OliveGreen}\itshape,
  stringstyle=\color{Orange},
  keepspaces,
  mathescape,
  escapeinside={(*@}{@*)}
}



%=======================================================================================================================

\newcommand{\Hl}[2]{%
    \tikz[overlay]\node[fill=#1,inner xsep=2pt,inner ysep=-1pt,anchor=text,rectangle,rounded corners=1mm] {\strut#2};%
    \phantom{#2}%
}

\renewcommand<>{\sout}[1]{%
    \alt#2{\beameroriginal{\sout}{#1}}{#1}
}

%=== Tikz ==============================================================================================================
\tikzstyle{rect}=[draw,rectangle,thick,on chain,font=\tiny,inner sep=1mm]

%== Title ==============================================================================================================
\title{ACM/SIGMOD 2019\\Programming Competition}
\date{\today}
\author{Immanuel L.\ Haffner}
\institute{Saarland University}


\begin{document}
\maketitle

\begin{frame}[t]{The Competition}
    \begin{block}{Task}
        In the competition, you have to
        \begin{enumerate}
            \item read an input file of records,
            \item sort them by their key in ascending order, and
            \item write the sorted sequence back to a file.
        \end{enumerate}
    \end{block}

    \pause
    \begin{block}{Evaluation}
        \begin{itemize}
            \item You submit a solution by uploading your code to the evaluation machine through the
                \emph{,,Dashboard''}, a web front end.

            \item Your submission is evaluated on four files of different sizes (10\,GB, 20\,GB, 40\,GB, and 100\,GB).
            \item You are being scored by the total time of execution.
        \end{itemize}
    \end{block}
\end{frame}

\begin{frame}[t]{The Competition}
    \begin{block}{Schedule}
        \footnotesize
        \begin{tabularx}{\textwidth}{l|X}
            March 05 & ,,Contest requirements specification and test data available.'' \\
            \onslide<2->{\color{Red} March 11 & Contest requirements specification available.}\\
            \onslide<2->{\color{Red} March 18 & Test data and sample solution available.}\\
            March 19 & Team registration begins. ,,Leaderboard available.'' \\
            \onslide<2->{\color{Red} March 22 & Leaderboard available.}\\
            April 30, 23:59 (UTC+2) & Final submission deadline. \\
            May 05 & Finalists notified.
        \end{tabularx}
    \end{block}
\end{frame}

\begin{frame}[t]{Detailed Task Description}
    \begin{block}{Input}
        Input file of $100$\,byte records:

        \begin{tikzpicture}
            \begin{scope}[start chain=going right,node distance=-\pgflinewidth]
                \node[rect] (k0) {$k_0$};
                \node[rect] {$k_1$};
                \node[rect] {$k_2$};
                \node[rect] {$k_3$};
                \node[rect] {$k_4$};
                \node[rect] {$k_5$};
                \node[rect] {$k_6$};
                \node[rect] {$k_7$};
                \node[rect] {$k_8$};
                \node[rect] {$k_9$};
                \node[rect,minimum width=50mm] (payload) {payload};
                \draw[decorate,decoration={brace,mirror,amplitude=10pt}]
                    (k0.south west) -- node[below=3mm] {\footnotesize\textbf{10 byte key}} (payload.south west);
                \draw[decorate,decoration={brace,mirror,amplitude=10pt}]
                    (payload.south west) -- node[below=3mm] {\footnotesize\textbf{90 byte payload}} (payload.south east);
            \end{scope}
        \end{tikzpicture}

        Input sizes between 10\,GB and 100\,GB means approx.\ 100\,M to 1\,B records.
    \end{block}

    \pause
    \begin{block}{Output}
        Write sorted records consecutively to output file.

        Result is validated using the \lstinline{valsort} tool.
    \end{block}

    \pause
    \begin{block}{Sorting}
        Sort the records in ascending order by their key $k = k_0 \circ k_1 \circ \dots \circ k_9$.
    \end{block}
\end{frame}

\begin{frame}{Hardware Specification}
    \begin{tabularx}{\textwidth}{l|X}
        Processor &
        \begin{minipage}[t][2.5cm]{\linewidth}
            2x Intel Xeon E5-2640 v4 @2.4GHz \\[1mm]
            \tiny
            \textbf{Performance:}
            \begin{itemize}
                \item 10 cores / 20 hyperthreads
                \item 2.40\,GHz base / 3.40\,GHz turbo
                \item 25\,MiB cache
                \item AVX-2
            \end{itemize}
        \end{minipage} \\\hline
        RAM & 30\,GiB \\\hline
        OS & Ubuntu 17.10 / Linux 4.13 \\\hline
        Disk &
        \begin{minipage}[t][2.5cm]{\linewidth}
            2x HPE EK001600GWEPH 1.6\,TB SATA-6GBPS \\[1mm]
            \tiny
            \textbf{Performance:}\\
            \begin{tabular}{ll}
                random reads & 61500 IOPS \\
                random writes & 54500 IOPS \\
                sequential reads & 500 MBPS \\
                sequential writes & 475 MBPS
            \end{tabular}
        \end{minipage}
    \end{tabularx}
\end{frame}

\begin{frame}[t]{Current State - I/O}
    So far, only the small and medium data sets are released.  All sorting can be done in-memory.

    \pause
    Current approach: Read $\rightarrow$ Sort $\rightarrow$ Write

    \pause
    \begin{block}{Reading}
        Read file in slabs of 256 disk blocks $\Rightarrow\ \sim$ 515\,MiB/s (exceeds specification :D)
    \end{block}

    \pause
    \begin{block}{Writing}
        Open the output file with \lstinline{mmap()}, allocate space with \lstinline{fallocate()} (with fallback to
        \lstinline{ftruncate()}), directly sort in the mapped output file.  Writing the file becomes a no-op, kernel
        ensures cached pages eventually make their way to disk. $\Rightarrow$ takes $\sim$ 0.5 sec, independent of file
        size (equals 40\,GiB/s and more)
    \end{block}
\end{frame}

\begin{frame}{Current State - Sorting}
    \begin{block}{Sorting}
        After reading and writing operates at the limits of the hardware, sorting is the final part to solve.\\
        Finding a fitting sorting algorithm for the given data is crucial!
    \end{block}
\end{frame}

\begin{frame}{Agenda - Sorting}
    \begin{enumerate}
        \item Implement, evaluate, and compare different sorting algorithms on real data and for different numbers of
            records.
        \item Design a hybrid approach that selects the best algorithm based on the size of the workload.
        \item Optimistically speaking, this hybrid approach should always be on par with the ,,best'' sorting algorithm
            for the given workload.
        \item Consider potentially skewed data in future workloads (large or xlarge).  Optimally, performance is
            independent of data distribution.
        \item Parallelize the algorithm.  Keep skew in mind, employ some kind of load balancing between worker threads.
    \end{enumerate}
\end{frame}

\begin{frame}{Agenda - Overlap Read and Sort}
    While reading the input, the CPU is mostly idle.  Exploit the computational power by piggy backing some work onto
    the reading.

    \pause
    \begin{itemize}[<+->]
        \item run generation with a tournament replacement selection
        \item histogram generation
        \item out-of-place partitioning
    \end{itemize}
\end{frame}

\begin{frame}{Agenda - External Memory Sorting}
    \centering
    \alert{\Huge ?}
\end{frame}

\begin{frame}
    %dummy
\end{frame}

\end{document}
